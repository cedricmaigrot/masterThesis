%%%%%%%%%%%%%%%%%%
%%%%    chapitre Methodes  %%%%
%%%%%%%%%%%%%%%%%%

\section{\'Etat de l'art et motivations}\label{TravauxRelatifs}

\textbf{Fouille du web social pour des perspectives médicales.}
La santé est un domaine où la fouille des réseaux sociaux permet d’envisager de réelles perspectives médicales.  Depuis 2013, 2 milliards d'utilisateurs sont actifs dans ces réseaux sociaux. Facebook est le plus peuplé avec 1,2 milliards d’individus, suivi par d'autres réseaux, dont Twitter avec 225 millions. Ces réseaux sont très utilisés pour partager des pensées, opinions et émotions avec ses proches. Au cours des cinq dernières années, il y a eu un intérêt croissant pour exploiter ces réseaux  comme un outil pour la santé publique, par exemple pour analyser la propagation de la grippe \cite{Sadilek12modelingspread}. \cite{ICWSM112880}  utilisent des modèles thématiques pour capturer les symptômes et les traitements possibles pour des maux évoqués sur Twitter afin de définir des mesures de santé publique. En examinant manuellement un grand nombre de tweets, \cite{Krieck11anew} ont montré que les symptômes auto-déclarés sont le signal le plus fiable pour prévoir l'apparition d'une maladie. 
%Ces études utilisent généralement les fréquences de termes spécifiques (e.g. H1N1, grippe, fièvre) et elles manquent parfois des indices qui peuvent être trouvés par des méthodes syntaxiques et sémantiques.
%D’un point de vue médical, la qualité des solutions proposées et la fiabilité des résultats est difficile à estimer, surtout quand les conclusions sont fondées sur un échantillon restreint d'exemples triés sur le volet.

\textbf{Fouille du web social pour la détection des maladies mentales et des personnes suicidaires.}
Dans le cadre de cette étude, nous nous concentrons sur le potentiel des réseaux sociaux pour surveiller les populations à risque suicidaire.

D’autres recherches existent portant sur les  maladies mentales en général. Dans ces applications, une personne est considérée comme à risque selon son utilisation des médias sociaux. Par exemple, le contenu de leurs tweets, les mises à jour de leurs statuts Facebook, sont utilisés pour classer en temps réel les personnes selon des niveaux de risque.   \cite{citeulike:12521251}  a démontré que des mises à jour de statut sur Facebook révèlent des symptômes d'épisodes dépressifs majeurs. \cite{park2012} ont trouvé des preuves que les gens utilisent Twitter pour poster sur leur dépression et leur traitement. Tous ces travaux soulignent le potentiel des médias sociaux comme une source de signaux pour la maladie mentale. 
Plus précisément, autour de la thématique du suicide, \cite{CashTPFB13}  analysent les messages d’adolescents sur MySpace.com afin de déterminer les sujets à risque (relations, santé mentale, toxicomanie/abus, méthode de suicide, déclarations sans contexte). \cite{Jashinsky14}  étudient les facteurs de risque de suicide par Twitter et trouve une forte corrélation entre les données Twitter dérivées et les données sur le suicide, ajustés selon l'âge. \cite{HomanSS13}  se concentrent sur un aspect particulier des tendances suicidaires, la détresse qui est un facteur de risque important dans le suicide et qui est observable à partir des textes de microblogs. Ils extraient des sujets d'intérêt pour les personnes à risque avec des techniques de modélisation non supervisées et ils mettent en place des méthodes de classification automatique pour identifier les personnes en détresse. \cite{Christensen} soulignent que des interventions sont possibles, bien que la validité, la faisabilité et la mise en œuvre reste incertaine car peu d'études ont été menées à ce jour dans des conditions réelles.

%Nous nous concentrons sur des données correspondant à des récits issus du réseau social Facebook et des données fournies par une association partenaire \textit{OADI} pour modéliser des concepts liés au suicide, différents niveaux de risque et un protocole de levée d'alerte. 
%L'originalité de notre approche est de prendre en compte non seulement les aspects thématiques mais également leur évolution temporelle via une adaptation d'un protocole de \textit{concept drift}. Nous invitons le lecteur a se reporter au document d'état de l'art (rédigé en avril) pour une bibliographie sur les méthodes de \textit{concept drift}.